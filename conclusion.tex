\chapter{Conclusión}

En este proyecto se logró un avance principal en el diseño principal de la estructura de cómo sería una consulta automatizada y qué tecnologías requiere para tal fin. Es suficiente decir que la base de conocimiento fue diseñada para persistir la información necesaria para consultas a los documentos normativos.

Sin embargo, cabe mencionar las dificultades encontradas en la estructuración del conocimiento. Mientras no existan formas estandarizadas de divulgar información semánticamente en los medios digitales, como lo es mediante internet, estos problemas seguirán existiendo. Esto debido a que, hasta la fecha, todavía se dependen de intervención humana para procesos que involucren comunicación o divulgación de información. Se espera que con este proyecto se enfatice la necesidad de estandarizar conocimiento para que una transformación digital de los procesos se hagan de manera óptima para las actividades que hacemos cotidianamente.

En cuanto a los objetivos, se considera que llevaban un avance acorde a lo planeado hasta la primera quincena de marzo. Pero hay que detallar el desvío de atención provocado por una subestimación en la planeación de la construcción de la base de conocimiento. Como se mencionó anteriormente, tenía que estructurarse adecuadamente para utilizarse en este proyecto. Se tuvo que recurrir al proceso manual de extracción a un formato intermedio. Aún así, se considera que las bases de diseño y funcionalidad de este proyecto están firmemente establecidas para que se continúe con las actividades del cronograma hasta concluir con una versión desplegable.